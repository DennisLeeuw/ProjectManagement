
In het technisch ontwerp maak je de vertaling van de functionaliteit naar een daadwerkelijk product of dienst die je gaat realiseren. Als de functie een tekstverwerker was dan komt er in het technisch ontwerp te staan dat er bijvoorbeeld LibreOffice Writer versie 7 wordt ge\"installeerd. Een technisch ontwerp gaat dus in op de technische details van een project en mag, of moet, de technische termen bevatten. Tekeningen gaan verder in op details, in plaats van wolkjes voor een netwerk worden de daadwerkelijke verbindingen getekend, etc.

Het is belangrijk om alle details te verwerken in het technisch ontwerp. Feitelijk moet het zo zijn dat als een ander het technisch ontwerp volgt dat hij dan een identieke installatie oplevert.

In het technisch ontwerp staat ook hoe de gekozen oplossing aansluit bij de bestaande infrastructuur en welke technische gevolgen dat heeft.

Het technisch ontwerp is bedoelt voor de technisch beheerders van de infrastructuur van de klant zodat deze weten wat ze krijgen en wat er van hun verwacht wordt, maar het is ook voor het projectteam zodat ze weten wat ze wanneer moeten opleveren.
