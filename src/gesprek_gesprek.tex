
Luister tijdens het gesprek goed. Mensen zeggen vaak meer dan ze denken. Let ook goed op lichaamshouding, daaruit blijkt al vaak hoe ze naar een onderwerp kijken. Zitten ze met de armen over elkaar dan staan ze niet zo open voor het onderwerp. Zitten ze ontspannen languit dan weet je dat ze een stuk positiever zijn. Kijk de spreker aan, een gezicht veel over de betekenis van de woorden.

Maak aantekeningen, maak aantekeningen en maak aantekeningen. We kunnen het niet vaak genoeg zeggen. Alleen met aantekeningen weet je ook nog volgende week wat er gezegd is. Je geheugen is niet zo goed dat je het allemaal kan onthouden.
