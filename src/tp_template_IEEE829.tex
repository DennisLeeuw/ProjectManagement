
Volgens de IEEE 829-2008 (Standard for Software and System Test Documentation) moet een testplan de volgende layout hebben:
\begin{itemize}
\item Pagina met daarop: Naam van het testplan en versienummer
\end{itemize}

\begin{enumerate}
\item Inleiding
\item Onderdelen die getest worden
\item Onderdelen die niet getest worden
\item Aanpak
\item Criteria voor pass of fail van een test
\item Criteria voor het onderbreken van een test en wanneer de test hervat wordt
\item Test uitkomsten
\item Test taken
\item Randvoorwaarden
\item Verantwoordelijkheden
\item Personeel en opleidingsbehoefte
\item Planning
\item Risico's
\item Goedkeuringen
\end{enumerate}

Het doel en wat we willen bereiken met de testen wordt beschreven in de inleiding. Hier moet ook vastliggen wat limieten zijn zowel qua geld als qua andere middelen zoals bijvoorbeeld benodigde tijd en personeel moet in de inleiding beschreven zijn.

De wel of niet te testen onderdelen en functies worden beschreven in de hoofdstukken Onderdelen die getest en Onderdelen die niet getest worden. In deze hoofstukken moeten referenties aanwezig zijn naar bijvoorbeeld het functioneel ontwerp of de behoefte analyse zodat duidelijk is dat de testen aansluiten of de behoefte van de organisatie.

In het hoofdstuk over de aanpak beschrijf hoe er getest gaat worden. Per test item is het noodzakelijk om vast te leggen welke data wordt gebruikt als input en output, wat de test procedures zijn per item en wat de volgorde is waarin de items getest moeten worden. Het is ook belangrijk om hier vast te leggen er nodig is om de testen te kunnen doen.

Bij de Test uitkomsten leg je vast wat er uit de test moet komen om succesvol te zijn, maar ook wat er zou moet gebeuren als een test faalt. Kan een volgende test gedaan worden, of is het een no-go criterium. De reden om het testen te staken of juist weer te hervatten zal per item vastgelegd moeten worden.

Wie er verantwoordelijk is voor welke taak op welk moment wordt vastgelegd in het Test taken hoofdstuk. Ook zal er per taak vastgelegd moeten worden wat ervoor nodig is en wat de doorlooptijd is. Dit hoofdstuk bevat dus de volledige planning voor de tests. Per taak zal er beschreven moeten worden wat er gedaan wordt zodat de verantwoordelijke voor test zijn taak goed kan uitvoeren. De verantwoordelijk zal ook een log van zijn testen bij moeten houden gedurende het testen zodat elke stap van de test naderhand gecontroleerd kan worden. Leg per test ook vast wat er opgeleverd moet worden naast de log om aan te tonen dat een test geslaagd is (of niet). De op te leveren gegevens zouden kunnen zijn, output data, screenshots, logs, etc.

Randvoorwaarden beschrijft welke zaken er aanwezig en op orde moeten zijn voordat aan een test begonnen kan worden. Het kan dan gaan om hardware zaken zoals spanning en netwerkaansluitingen, maar ook om personele aanwezigheid of toegang tot ruimtes. Beschrijf hier ook wat er nodig is als er een speciale testomgeving gebouwd moet worden.

Verantwoordelijkheden bevat de taakomschrijvingen van het testteam, ervan uitgaande dat er meer dan 1 persoon betrokken is bij de test.

Als er mensen opgeleid moeten worden of speciaal geinstrueerd moeten worden voordat een test plaats kan vinden dan wordt dat beschreven in Personeel- en opleidingsbehoefte.

Een overzicht van welke test op welke datum en op welk tijdstip moet worden uitgevoerd door wie wordt vastgelegd in de Planning. Gebruik hiervoor een applicatie om planningen te maken zoals ProjectLibre.

Aan testen kunnen risico's verbonden zijn, zeker als bij de testen productie systemen betrokken zijn. Het is belangrijk belangrijk dat deze risico's beschreven zijn en dat er vastgelegd is hoe de risico's geminimaliseerd kunnen worden en wat te doen als er iets fout mocht gaan.

Uiteindelijk moet je toestemming krijgen om de testen te mogen uitvoeren. In Goedkeuringen staan de namen van de personen die moeten tekenen om goedkeuring te geven aan de testen. Pas als alle handtekeningen binnen zijn kan er aan het testen begonnen worden.
