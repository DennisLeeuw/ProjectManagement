
\renewcommand{\labelenumii}{\theenumii}
\renewcommand{\theenumii}{\theenumi.\arabic{enumii}.}
\renewcommand{\theenumiii}{\theenumii\arabic{enumiii}}

Bij het examen gebruikt Stichting Praktijkleren een andere variant op het testplan. Deze heeft de volgende structuur:
\begin{enumerate}
\item Inleiding
\item Overzicht van te testen functies
	\begin{enumerate}
	\item Functie x
		\begin{enumerate}
		\item Te testen functionaliteit
		\item Test scenario
		\item Test input
		\item Verwachte werking en output
		\item Werkelijke werking en output
		\item Conclusie test
		\end{enumerate}
	\end{enumerate}
\item Test omgeving
	\begin{enumerate}
	\item Overzicht test omgeving
	\item Planning opzet test omgeving
	\end{enumerate}
\item Planning testactiviteiten
\item Eind conclusie testen
\end{enumerate}

Het testplan en het testrapport zijn in deze opzet in \'e\'en document verwerkt.

In het Overzicht van te testen functies beschrijf je per functie hoe de test eruit moet zien en is er tevens ruimte om de uitkomsten van de test te verwerken. De x bij item 1.1 moet vervangen worden door de naam van de functie die getest gaat worden.

Bij Test omgeving beschrijf je wat er nodig is om de test te kunnen uitvoeren. Dit is het materiaal dat nodig is, de software en de mensen. Het bevat dus ook een planning die nodig is om de testomgeving te realizeren.

Om de testen die beschreven zijn te kunnen uitvoeren moet er een planning gemaakt worden waarin beschreven wordt wie welke testen uitvoert. Het kan belangrijk zijn om de testen in een bepaalde volgorde te doen, als dat zo is moet dat uit de planning blijken. De planning bevat de startdatum en het tijdstip waarop de test uitgevoerd wordt. Daarnaast legt het ook vaak vast hoelang een test duurt of voor wanneer hij klaar moet zijn.

Het laatste onderdeel van dit testplan is de Eind conclusie testen. Bij de conclusie beschrijf je of de totale test een uitkomst die voldoende was om het project als succesvol te beschouwen, of je geeft aan wat er niet goed gegaan is en wat er nodig is om verdere stappen te ondernemen.

