\begin{itemize}
\item Behoefte analyse omzetten naar functies in een informatie systeem
\item Gevolgen van de functies bepalen
\item Globale planning opstellen
\item Globale kostenraming opstellen
\item Afstemmen met opdrachtgever en collega's
\item Functioneel ontwerp vastleggen
\item Presentatie aan de belanghebbenden in de organisatie
\item Goedkeuring functioneel ontwerp
\end{itemize}

De eerste stap is het vertalen van de eisen en wensen naar functies. Een functie is een taak die uitgevoerd moet worden. Als de eis is dat er een tekst verwerkt moet kunnen worden dan kan een functie zijn het installeren van een tekstverwerker.

Door te bepalen wat de gevolgen zijn van een functie zet je de risico's voor een organisatie op een rij. Het installeren van een tekstverwerker betekent bijvoorbeeld dat er extra diskruimte in beslag genomen gaat worden en dat er tijdens het gebruik meer geheugen gebruikt gaat worden, maar het betekent bijvoorbeeld ook dat gebruikers hun documenten ergens moeten kunnen opslaan. Dit onderzoek naar de gevolgen kan behoorlijk complex zijn. Functies kunnen ook een invloed hebben op gebruikers (bijvoorbeeld training) of beheer (bijvoorbeeld up-to-date houden).

Op het moment dat je weet wat er allemaal gebeuren moet en wat dat voor gevolgen heeft kunnen een globale planning opzetten. Een planning is een lijstje dat aangeeft wat in welke volgorde gebeuren moet om tot de juiste oplossing te komen.

Een gevolg van een planning en een overzicht met wat er gebeuren moet is dat je een eerste kostenoverzicht kan maken. Je weet immers wat er aangeschaft moet worden (bijvoorbeeld hardware, of software) en wat er aan uren nodig zijn om tot een eindproduct te komen. Als het functioneel ontwerp goedgekeurd is kan je een offerte maken en ligt de prijs van het project definitief vast.
