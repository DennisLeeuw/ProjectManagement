
Tijd is over het algemeen kostbaar, toch hebben we tijd nodig van mensen om beter inzicht te krijgen in de eisen en wensen van een organisatie. Om de hoeveelheid tijd die je van iemand vraagt te beperken is het noodzakelijk om je goed voor te bereiden. Zorg dat je vragen op papier staan, bedenkt wat de mogelijke antwoorden zouden kunnen zijn en bedenk dat je dan nog verder zou willen weten.

Weet aan wie je de vragen wil stellen. Wat is hun functie? En tot welke details zouden ze je antwoord kunnen geven? Een directeur weet vaak in grote lijnen wel bij welke afdeling je je vraag kwijt kan en die weet ook wat voor uitstraling hij wil hebben, maar die heeft vaak geen idee van welke server wat doet. Een systeembeheerder daarentegen weet precies welke server wat doet, maar heeft waarschijnlijk geen idee wat de bedrijfskleuren zijn en welk font er op een website gebruikt moet worden. Zo heeft iedereen binnen een organistatie zijn kennis en kunde, stel je vragen daarop af en kies de te interviewen mensen zorgvuldig.

Maak van elk gesprek dat je gevoerd hebt een gespreksverslag. Deel dit met de geinterviewde en vraag om een bevestiging. Alleen zo weet je zeker dat je de juiste informatie hebt opgeschreven en de persoon goed begrepen hebt. Fouten in het begin van het proces hebben grote gevolgen aan het eind. Als je twijfelt of je iets goed hebt genoteerd, vraag het dan opnieuw. Controle van je data is essentieel. Een niet gestelde vraag zorgt vrijwel zeker voor een niet goed opgeleverd project.
