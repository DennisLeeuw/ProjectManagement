
Voordat je een informatiesysteem gaat inrichten is het van belang om te weten wat een klant/gebruiker wil. De eerste taak die dan ook uitgevoerd moet worden is het verzamelen van informatie om te weten te komen wat de wensen van de gebruiker zijn. De belangrijkste twee vragen daarbij zijn de wat en de waarom vraag.

Wat een gebruiker wil is meestal al snel duidelijk. Een gebruiker wil bijvoorbeeld een website voor zijn winkel. Bij de wat vraag moeten ook de verwachtingen van de klant naar voren komen. Het maakt nogal wat uit of de klant 200 artikelen in zijn shop wil opnemen of 2.000.000. En het maakt voor de infrastructuur veel uit of hij 200 bestellingen per maand of 2.000.000 verwacht. Bij de waarom vraag moeten dus de doelen van de gebruiker naar boven komen.

Waarom een gebruiker het wil is een stuk lastiger, maar maakt uiteindelijk wel duidelijker waaraan het eindproduct moet voldoen. Een klant die een autoshowroom binnenkomt met zijn zwangere vrouw opzoek naar een nieuwe auto, daarvan is duidelijk dat ze geen behoefte hebben aan een cabriolet voor 2 personen. Waarom? Omdat er een kind bij komt, maar dit is een aanname en het is dus goed om de waarom vraag toch te stellen. Want misschien hebben ze al een 4-persoonsauto en zoeken ze een cabrio om er in de weekenden samen mee weg te kunnen als de baby bij opa en oma is. Zo zie je dat het belangrijk is om vragen altijd te stellen. Vragen stellen is niet dom, een vraag niet stellen en iets aannemen kan tot een verkeerd product en dus een ontevreden klant leiden.

