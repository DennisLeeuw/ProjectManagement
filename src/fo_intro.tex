
In het functioneel ontwerp leg je in niet technische termen vast hoe het project eruit gaat zien. Dit document is over het algemeen voor management en zij moeten het document dan ook kunnen snappen. De opdrachtgever (klant) en de opdrachtnemer moeten hetzelfde beeld hebben van de uit te voeren opdracht.

Mocht je in een functioneel ontwerp het gebruik van technische termen niet kunnen vermijden zorg er dan voor dat je die termen uitlegt. Beter is het vaak om een plaatje te gebruiken, voor niet techneuten is dat vaak duidelijker dan tekst.

Flowcharts, stroomdiagrammen en globale netwerktekeningen zijn goede oplossingen om techniek uit te leggen aan niet techneuten.

Een functioneel ontwerp kan naast een document ook vaak leiden tot een presentatie waarin, door de opdrachtnemer, aan de opdrachtgever uitgelegd wordt wat het project betekent voor de organisatie en wat voor gevolgen het heeft.
