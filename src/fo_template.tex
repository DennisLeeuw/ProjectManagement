
\begin{enumerate}
\item Inleiding
\item Achtergrond informatie
\item Aannemer
\item Opdrachtgever
\item Probleemstelling
\item Functionele eisen
\item Nieuwe situatie
\item Globale planning
\item Globale kostenraming
\end{enumerate}

Bij de inleiding geef je aan dat dit het functionele ontwerp is voor het project met een minimale omschrijving van waar het project over gaat. Ook geef je hierin aan voor wie dit document bedoelt is.

In de achtergrond informatie beschrijf je het bedrijf of de organisatie waarvoor het project wordt uitgevoerd. Wat voor bedrijf is het? Hoeveel mensen werken er? Wat is de huidige situatie? en waarom gaan we dit project doen?

De aannemer is het bedrijf, organisatie of persoon die de opdracht gaat uitvoeren. Naast de naam en contactgegevens kan hier een stukje omschrijving geplaatst worden over de aannemer.

De opdrachtgever is het bedrijf of de afdeling binnen een bedrijf waarvoor we de opdracht uitvoeren. Neem hierin ook de contact gegevens op van de mensen binnen de organisatie die belangrijk zijn voor het slagen van het project.

De probleemstelling bevat een uitgebreidde omschrijving van welk probleem het project moet oplossen.

De functionele eisen leggen vast wat er moet gebeuren en hier kan de MoSCoW-methode weer gebruikt worden

Tot slot leg je vast in de nieuwe situatie wat het eindresultaat zal zijn en wat dat voor gevolgen heeft voor de organistatie.

Voor het management volgen hierna nog een grove planning, zodat duidelijk is hoelang een project gaat duren (en waarom) en een globaal overzicht van de kosten.
