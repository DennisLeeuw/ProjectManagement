De uitkomsten uit de gedane testen moeten vastgelegd worden in een testrapport. Dit rapport wordt voorgelegd aan de organisatie ter goedkeuring. Als de organisatie akkoord gaat met de uitkomsten van de testen kan het systeem in productie genomen worden.

Worden de uitkomsten niet geaccepteerd, dan zal men moeten kijken wat de vervolgstappen worden:
\begin{itemize}
\item Verbeteringen doorvoeren om tot andere testuitkomsten te komen
\item Wijzigingen in de testen doorvoeren omdat de gekozen testen niet correct zijn
\item Terug naar de tekentafel om tot een ander ontwerp te komen dat wel aan de eisen van de opdrachtgever voldoet
\item Het project in zijn geheel afblazen
\end{itemize}
