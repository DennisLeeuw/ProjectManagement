
Niet alles is even belangrijk in een project. Ook kunnen geld- of tijdslimieten ervoor zorgen dat zaken beter niet opgenomen kunnen worden. Het is dus van belang dat er vast gelegd gaat worden wat we wel en niet doen, en wat we doen als het kan. Binnen het bedrijfsleven is daarvoor de MoSCoW-methode ontstaan. MoSCoW is een afkorting waarbij de o's niet betekenen. Het gaat alleen om de letters MSCW

\begin{itemize}
\item M - Must have (moeten we hebben)
\item S - Should have (kunnen we eigenlijk niet zonder)
\item C - Could have (doen we als tijd en geld het toelaat)
\item W - Won't have (doen we niet)
\end{itemize}

Het is vaak moeilijk om de eisen en wensen in de juiste categorie te plaatsen.

De Must haves zijn echte eisen. In deze categorie komen zaken te staan die we moeten doen. Als deze zaken niet gedaan worden hebben we geen werkend product.

De Should haves zijn wensen die we moeten doen. Zonder deze wensen werkt het product wel, maar het voldoet dan niet aan bijvoorbeeld bedrijfs- of beveiligingsrichtlijnen. Als we niet aan deze wensen voldaan hebben kan het project niet in productie opgenomen worden

De Could haves zijn wensen die we willen doen, maar die niet verhinderen dat een project in productie gaat. Het zou mooi zijn als we ze kunnen vervullen en we zullen in het project ook ons best doen om het te doen, maar als ze niet gedaan zijn kan het project wel vast in productie gaan en worden ze op een later tijdstip alsnog gedaan.

Won't haves zijn zaken die we niet doen binnen het project. Het legt dus de grenzen vast. Dit is vaak het lastigste onderdeel om goed in te vullen.
