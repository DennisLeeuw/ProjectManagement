De aanleiding voor een project kan op verschillende manieren ontstaan. Er kan een probleem zijn dat opgelost moet worden, maar het kan ook een compleet nieuw systeem zijn dat nieuwe functionaliteit oplevert, dan komt het dus voort uit een wens. Bij beide is het van belang om te weten wat het einddoel moet zijn, kortom welke service moet er worden opgeleverd. Om deze informatie bovenwater te krijgen begin je meestal bij de wens of het probleem zoals deze op tafel is komen te liggen. In de rest van het document gaan we ervan uit dat we bij projectmatig werken een probleem oplossen.

Over het algemeen is de binnengekomen vraag niet voldoende om op basis daarvan een oplossing te gaan bouwen. Het is dus van belang dat er informatie verzameld wordt om duidelijker te krijgen wat er precies moet komen en aan welke randvoorwaarden er voldaan moet worden. Deze fase in het project heet de behoefte analyse omdat je gaat verzamelen wat de behoefte, wensen en eisen, zijn van de klant.

De tweede stap is het beschrijven wat de functionelen eisen zijn waaraan een oplossing moet voldoen. Dit is vaak in termen die door een leek begrepen kunnen worden.

De derde stap is het maken van een technisch ontwerp. In een technisch ontwerp komen alle technische details te staan die vast leggen wat je gaat opleveren.

De laatste fase is het testen van de oplossing om te zien of deze voldoet aan de eisen en wensen van de klant. Alle punten uit de hoefte analyse en het functioneel ontwerp zullen hier getest moeten worden en goedgekeurd door de opdrachtgever. Het is dus van essentieel belang dat de opdrachtgever de behoefte analyse en het functioneelontwerp heeft begrepen en goedgekeurd, anders kun je noot in deze fase laten zien dat je aan zijn vraag hebt voldaan.
