Wat ook helder moet zijn voordat we aan een project beginnen is het doel. Een gebruiker heeft vaak de neiging om een product neer te leggen als oplossing. Ik wil Photoshop! Bij doorvragen komen er meestal eisen en wensen op tafel waaraan het gekozen product niet of niet optimaal voldoet. In de behoefte analyse is het dan ook van belang om nog niet in oplossingen te denken maar de informatie te verzamelen waaraan de oplossing moet voldoen. In het geval van Photoshop zou je dus moeten vaststellen wat de gebruiker wil doen:
\begin{itemize}
\item Moet foto's van het formaat TIFF, JPEG en DICOM kunnen lezen
\item Moet deze foto's kunnen roteren tussen 0 en 360 graden
\item Moet META-data aan foto's mee kunnen geven in IPTC en DICOM
\end{itemize}
Op deze manier maak je en lijst met wensen en eisen. Het is daarbij van belang om vast te stellen wat er in de eindoplossing moet zitten (eis) en wat fijn zou zijn als het erin zit (wens).

Een ander belangrijkpunt zijn de randvoorwaarden. Kortom binnen welke perken moetn we blijven qua budget, personen die ingezet mogen worden op het project of de tijdsduur, is er bijvoorbeeld een deadline waarvoor het project af moet zijn.

Het laatste dat we tijdens de behoefte analyse moeten inventariseren is wat de kansen en risico's zijn. Is het project bijvoorbeeld afhankelijk van een ander project, of kunnen we aansluiten bij een al bestaand project? Let ook op of er conflicten zijn tussen de verschillende wensen en eisen en of het geheel wel past binnen de bestaande infrastructuur. Het zou bijvoorbeeld zo kunnen zijn dat uit de wensen en eisen blijkt dat er een Linux met Apache-webserver moet komen, maar dat er binnen de organisatie alleen Windows gebruikt wordt. Het is dan de vraag of er geinvesteerd gaat worden in de opbouw van kennis om een Linux server te beheren, of dat er toch gekozen gaat worden voor een product dat kan draaien op een IIS server.
