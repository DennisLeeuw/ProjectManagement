
\begin{itemize}
\item omzetten van de functies in het functioneel ontwerp naar daadwerkelijke producten of diensten
\item Koppeling met bestaande systemen beschrijven
	\begin{itemize}
	\item gedetaileerde beschrijving van hoe de producten of diensten aansluiten op de bestaande infrastructuur
	\item beschrijving van wat er nodig is om de producten of diensten te kunnen plaatsen binnen de huidige infrastructuur
	\end{itemize}
\item Selecteren van systemen, producten, software
	\begin{itemize}
	\item welke producten of diensten worden er geleverd
	\end{itemize}
\item Afstemming met interne (klant) en eigen collega's
	\begin{itemize}
	\item beschrijving van de wijzigingen in de infrastructuur of in de dienstverlening en de impact op de organisatie (beheer, support)
	\end{itemize}
\item 
\end{itemize}

De eerste stap is het technisch beschrijven van de producten of diensten die worden geleverd voor elke functie uit het functioneel ontwerp. De installatie van een tekstverwerker wordt daadwerkelijk gerealiseerd met LibreOffice Writer versie 7.

Op basis van de te leveren diensten of producten moeten er systemen uitgezocht worden die de technische diensten gaan waar maken. Voor de opslag van de gebruikers documenten moet er op de fileserver per gebruiker een directory aangemaakt worden om bestanden op te kunnen slaan. Dit heeft tot gevolg dat het RAID systeem van de server met twee extra harddisks moet worden uitgebreid. De harddisks moeten 10k rpm harddisks zijn die passen binnen de bestaande RAID set met 600GB schijven gebaseerd op een Dell PERC H710 Controler.

Zorg ook dat bijvoorbeeld IP-adressen, systeemnamen en ook lokaties in kasten bekend zijn. Zijn er voldoende netwerkpoorten en spanningsaansluitingen, etc. Denk daarbij ook aan printers, werkplekken en randapparatuure.

Naast LibreOffice Writer 7 moeten er dan bijvoorbeeld ook Dell 600GB 10K RPM SAS 12Gbps schijven worden aangeschaft. Dit komt in een lijst te staan met te leveren producten.

Tot slot is het van belang dat iedereen, zowel de beheerders van de klant als de leden van het projectteam weten wat er wanneer van ze wordt verwacht. Wanneer kunnen bijvoorbeeld de harddisks in de server geplaatst worden en welke personen van de klant moeten daarbij aanwezig zijn voor toegang tot de serverruimte. Let er ook op dat na de implementatie de organisatie het systeem moet kunnen beheren en supporten, dus ook daarvoor moeten de zaken op orde zijn (kennis, documentatie).
