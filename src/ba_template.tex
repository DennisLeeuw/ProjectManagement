\begin{enumerate}
\item Inleiding
\item Opdracht
\item Aanleiding
\item Knelpunten
\item Eisen/wensen
	\begin{enumerate}
	\item Must haves
	\item Should haves
	\item Could haves
	\item Won't haves
	\end{enumerate}
\item Uit te voeren werkzaamheden
\end{enumerate}

In de inleiding leg je vast dat dit de behoefte analyse is en waar het onderwerp globaal over gaat.

In de opdracht beschrijf je wat de gekregen opdracht is in je eigen woorden en leg je het doel van het project vast.

Bij de aanleiding leg je vast waarom dit project wordt gedaan en wie er belang heeft bij het project. Dit legt onder andere vast wie er betrokkken moeten worden bij het project om tot een goed eindresultaat te komen.

Bij knelpunten leg je vast wat de problemen zijn waarvoor dit project de oplossing is. Het is belangrijk om aan het einde van het project te controleren of alle hier genoemde problemen ook daadwerkelijk opgelost zijn. Het kan heel handig zijn om hier een vinklijstje te maken van de bestaande problemen.

Bij uit te voeren werkzaameheden leg je een grof stappen plan vast. Je gaat niet in details maar geeft aan wat in welke volgorde moet gebeuren. Dit stappen plan legt ook vast wie er van de organistatie bij welke stap betrokken moet zijn. De organisatie moet hier zijn goedkeuring aan geven, want het betekent dat mensen de tijd moeten hebben om aanwezig te zijn, of om actie te ondernemen.
