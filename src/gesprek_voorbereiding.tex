
Voordat je een gesprek in gaat moet je je goed voorbereiden. Zorg dat je weet dat het onderwerp is dat besproken gaat worden. Als er een agenda is gemaakt, zorg er dan voor dat je deze bestudeerd hebt en weet wat er aan de orde gaat komen. Zijn er andere stukken die je vast lezen kan, zoals bijvoorbeeld het verslag van de vorige keer.

Maak aantekeningen bij de gelezen stukken of bij de agenda als je iets in wil brengen of op wil merken. Als het heel belangrijk is laat het dan opnemen op de agenda zodat iedereen weet dat het besproken gaat worden en anderen zich er ook op kunnen voorbereiden.
