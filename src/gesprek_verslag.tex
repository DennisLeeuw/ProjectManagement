
In een gespreksverslag moeten een aantal zaken terug komen:
\begin{itemize}
\item Wanneer vond het sprek plaats (datum)
\item Wie waren er aanwezig
\item Wat waren de onderwerpen
\item Wat was de aanleiding voor het gesprek
\item Wat zijn de gemaakte afspraken
\end{itemize}

Per besproken onderwerp is het goed om beknopt weer te geven wat er besproken is. Het hoeft niet in de vorm "Jan zei: bladiebla; Piet zei: dieblahdie", het mag ook in de vorm van "We hebben besproken dat blahdieblahdieblah".

Een afsprakenlijst of een actiepuntenlijst kan een aantal elementen bevatten:
\begin{center}
\begin{tabular}{ | c | c | c | c | c | }
\hline
 Datum & Afspraak & Wie & Gereed & Opmerkingen \\ 
\hline
\end{tabular}
\end{center}

Datum bevat de datum dat de afspraak gemaakt is en Gereed de datum waarop het gereed moet zijn. In de kolom met Afspraak komt te staan wat er afgesproken is en bij Wie de naam of namen van de personen die verantwoordelijk zijn voor het nakomen van de afspraak.

