\documentclass[a4paper,12pt,twoside,openright,titlepage]{book}

%Additional packages
\usepackage[ascii]{inputenc}
\usepackage[T1]{fontenc}
\usepackage[dutch,english]{babel}
\usepackage{imakeidx}
\usepackage{syntonly}
\usepackage[official]{eurosym}
%\usepackage[graphicx]
\usepackage{graphicx}
\graphicspath{ {./images/} }
\usepackage{wrapfig}
\usepackage{float}
\usepackage{xurl}
\usepackage{hyperref}
\hypersetup{colorlinks=true, linkcolor=blue, citecolor=blue, filecolor=blue, urlcolor=blue, pdftitle=, pdfauthor=, pdfsubject=, pdfkeywords=}
\usepackage{tabularx}
\usepackage{scrextend}
\addtokomafont{labelinglabel}{\sffamily}
\usepackage{listings}
\usepackage{adjustbox}
\usepackage{color}

% Define colors
\definecolor{ashgrey}{rgb}{0.7, 0.75, 0.71}

% Listing style
\lstset{
  backgroundcolor=\color{ashgrey},   % choose the background color; you must add \usepackage{color} or \usepackage{xcolor}; should come as last argument
  basicstyle=\footnotesize,        % the size of the fonts that are used for the code
  breakatwhitespace=false,         % sets if automatic breaks should only happen at whitespace
  breaklines=true,                 % sets automatic line breaking
  extendedchars=true,              % lets you use non-ASCII characters; for 8-bits encodings only, does not work with UTF-8
  frame=single,	                   % adds a frame around the code
  keepspaces=true,                 % keeps spaces in text, useful for keeping indentation of code (possibly needs columns=flexible)
  rulecolor=\color{black},         % if not set, the frame-color may be changed on line-breaks within not-black text (e.g. comments (green here))
  showspaces=false,                % show spaces everywhere adding particular underscores; it overrides 'showstringspaces'
}

% Uncomment for production
% \syntaxonly

% Style
\pagestyle{headings}

% Turn on indexing
\makeindex[intoc]

% Define document
\author{D. Leeuw, Wim Onrust}
\title{Project Management}
\date{\today\\v.1.1.1}

\begin{document}
\selectlanguage{dutch}

\maketitle

\copyright\ 2022 Dennis Leeuw, Wim Onrust\\

\begin{figure}
\includegraphics[width=0.3\textwidth]{CC-BY-SA-NC.png}
\end{figure}

\bigskip

\input{src/licentie}

%%%%%%%%%%%%%%%%%%%
%%% Introductie %%%
%%%%%%%%%%%%%%%%%%%

\frontmatter
%\chapter{Over dit Document}
%\input{src/OverDitDocument}
%\input{src/DocChanges_GUI}

%%%%%%%%%%%%%%%%%
%%% De inhoud %%%
%%%%%%%%%%%%%%%%%
\tableofcontents

\mainmatter

\chapter{Project Management}
In de opdracht van de overheid wat een MBO student moet kunnen als hij van school komt staat dat deze instaat moet zijn een informatievoorziening te ontwerpen. Een informatievoorziening kan een webserver zijn, maar ook een webserver met database of een compleet netwerk met authenticatiesystemen, opslagsystemen en gebruikerssystemen. Een informatievoorziening of een informatiesysteem is een samenhangend geheel van apparaten die data verwerken. En dat moet je als MBO-er dus kunnen ontwerpen.

Een informatievoorziening is dus een systeem dat gebouwd moet worden van begin tot eind. Vanuit de ITIL weten we dat als we een actie hebben met een duidelijk begin en een duidelijk einde dat we dan spreken van een project. Om een project in goede banen te leiden moeten we een project beheren of managen. Vandaar dat dit document gaat over project management. Je leert welke stappen je moet doorlopen om een project van begin tot eind te doorlopen zodat de wensen van een klant of gebruiker gerealizeerd worden en er een gelukkige klant is.

Om te zorgen dat een informatiesysteem opgeleverd wordt volgens de eisen en wensen van de klant is het van belang om een aantal stappen te doorlopen. De eerste stap is dat het duidelijk moet zijn wat de klant wil. Daarna kan er een plan gemaakt worden en kunnen we aan de bouw beginnen. Tot slot moet het gebouwde systeem opgeleverd worden en hopen we dat we een gelukkige klant hebben. Dit zijn de stappen die in bijna elk project doorlopen moeten worden om te komen tot een werkende oplossing. Projectmanagement (ook wel projectbeheer) is het beheersen van projecten. Het is de manier waarop projecten georganiseerd, voorbereid, gepland, uitgevoerd en afgerond worden. Er is een functie voor deze taak, dat zijn de zogenaamde projectmanagers. Deze managers doen niets anders dan projecten, dat werkt goed binnen grote organisaties, binnen kleine(re) organisaties wordt het projectmanagement vaak gedaan door degene die het meeste belang heeft bij het project.

In de loop van de tijd zijn er verschillende visies ontstaan op hoe al deze stappen doorlopen moeten worden. Algemene projectmanagement methodes zijn:
\begin{itemize}
\item EVO
\item IPMA
\item OPEN
\item PRINCE2
\item PMBOK
\item PM$^2$
\item Het PM3-model
\end{itemize}

In de ICT zijn verschillende methodes ontstaan voornamelijk rond software-ontwikkeling die veel gebruikt worden:
\begin{itemize}
\item Agile
\item DSDM
\item RUP
\item Scrum
\item SDM
\item Watervalmethode
\end{itemize}

De verschillende methodes zijn soms vrij beschikbaar voor iedereen, soms zijn ze gecertificeerd, wat inhoudt dat gebruikers een certificaat/diploma moeten halen om zichzelf 'gecertificeerd' projectleider te mogen noemen. De certificaten worden in de regel verkregen door (tegen betaling) een examen te doen bij een instituut dat de desbetreffende methode heeft uitgevonden. Dit certificaat moet in de regel om de zoveel jaar vernieuwd worden.

De waterval methode is behoorlijk formeel, met veel documentatie en is niet erg flexibel, als de plannen vastliggen is er geen wijziging meer mogelijk tijdens het bouwproces. Bij de waterval methode worden alle stappen \'e\'en voor \'e\'en doorlopen en wordt zo naar de eindoplossing toegewerkt.

De agile en scrum zijn wat losser, met minder documentatie en iets flexibeler, er kunnen wijzigingen blijven plaats vinden gedurende het bouw proces. Bij beide methodes wordt \'e\'en stap doorlopen en deze wordt geanalyzeerd en er wordt gekeken of we nog op het doel af gaan, of dat door nieuwe inzichten het doel verschoven is. Zodat we moeten bijsturen voordat we de volgende stap zetten. Het is dus een meer iteratief proces.

Welke methode het beste is hangt af van het product of de oplossing die binnen het project gerealizeerd moet worden en het hangt heel vaak ook af van de organisatie waarvoor je werkt. In dit document zullen we alle stappen op een waterval manier doorlopen. Er zijn een aantal redenen waarom er gekozen is voor de watervalmethode. De eerste is dat we waterval methode duidelijk afgebakende stappen kent. De tweede reden is dat het terug komt op je examen. En de laatste reden is dat als je de watervalmethode kent, je makkelijk andere methodes ook begrijpt.

\section{Aanvang}
De aanleiding voor een project kan op verschillende manieren ontstaan. Er kan een probleem zijn dat opgelost moet worden, maar het kan ook een compleet nieuw systeem zijn dat nieuwe functionaliteit oplevert, dan komt het dus voort uit een wens. Bij beide is het van belang om te weten wat het einddoel moet zijn, kortom welke service moet er worden opgeleverd. Om deze informatie bovenwater te krijgen begin je meestal bij de wens of het probleem zoals deze op tafel is komen te liggen. In de rest van het document gaan we ervan uit dat we bij projectmatig werken een probleem oplossen.

Over het algemeen is de binnengekomen vraag niet voldoende om op basis daarvan een oplossing te gaan bouwen. Het is dus van belang dat er informatie verzameld wordt om duidelijker te krijgen wat er precies moet komen en aan welke randvoorwaarden er voldaan moet worden. Deze fase in het project heet de behoefte analyse omdat je gaat verzamelen wat de behoefte, wensen en eisen, zijn van de klant.

De tweede stap is het beschrijven wat de functionelen eisen zijn waaraan een oplossing moet voldoen. Dit is vaak in termen die door een leek begrepen kunnen worden.

De derde stap is het maken van een technisch ontwerp. In een technisch ontwerp komen alle technische details te staan die vast leggen wat je gaat opleveren.

De laatste fase is het testen van de oplossing om te zien of deze voldoet aan de eisen en wensen van de klant. Alle punten uit de hoefte analyse en het functioneel ontwerp zullen hier getest moeten worden en goedgekeurd door de opdrachtgever. Het is dus van essentieel belang dat de opdrachtgever de behoefte analyse en het functioneelontwerp heeft begrepen en goedgekeurd, anders kun je noot in deze fase laten zien dat je aan zijn vraag hebt voldaan.


\chapter{Gespreksverslag}

Tijdens je werkzaamheden zal je veel mensen spreken, soms even in de wandelgangen en soms formeel omdat je een afspraak met die persoon hebt gemaakt. Als een gesprek een wat formeler karakter heeft, bijvoorbeeld in interview of een vergadering, dan is het handig om een gespreksverslag te, laten, maken van het gesprek. We gaan er bij dit onderwerp vanuit dat jij het verslag moet maken, dus ook dat onderwerp zal hier aan de orde komen. Verder behandelen belangrijke zaken die je bij elk gesprek van pas kunnen komen.

\section{Voorbereiding}

Voordat je een gesprek in gaat moet je je goed voorbereiden. Zorg dat je weet dat het onderwerp is dat besproken gaat worden. Als er een agenda is gemaakt, zorg er dan voor dat je deze bestudeerd hebt en weet wat er aan de orde gaat komen. Zijn er andere stukken die je vast lezen kan, zoals bijvoorbeeld het verslag van de vorige keer.

Maak aantekeningen bij de gelezen stukken of bij de agenda als je iets in wil brengen of op wil merken. Als het heel belangrijk is laat het dan opnemen op de agenda zodat iedereen weet dat het besproken gaat worden en anderen zich er ook op kunnen voorbereiden.

\section{Tijdens het gesprek}

Luister tijdens het gesprek goed. Mensen zeggen vaak meer dan ze denken. Let ook goed op lichaamshouding, daaruit blijkt al vaak hoe ze naar een onderwerp kijken. Zitten ze met de armen over elkaar dan staan ze niet zo open voor het onderwerp. Zitten ze ontspannen languit dan weet je dat ze een stuk positiever zijn. Kijk de spreker aan, een gezicht veel over de betekenis van de woorden.

Maak aantekeningen, maak aantekeningen en maak aantekeningen. We kunnen het niet vaak genoeg zeggen. Alleen met aantekeningen weet je ook nog volgende week wat er gezegd is. Je geheugen is niet zo goed dat je het allemaal kan onthouden.

\section{Gespreksverslag}

In een gespreksverslag moeten een aantal zaken terug komen:
\begin{itemize}
\item Wanneer vond het sprek plaats (datum)
\item Wie waren er aanwezig
\item Wat waren de onderwerpen
\item Wat was de aanleiding voor het gesprek
\item Wat zijn de gemaakte afspraken
\end{itemize}

Per besproken onderwerp is het goed om beknopt weer te geven wat er besproken is. Het hoeft niet in de vorm "Jan zei: bladiebla; Piet zei: dieblahdie", het mag ook in de vorm van "We hebben besproken dat blahdieblahdieblah".

Een afsprakenlijst of een actiepuntenlijst kan een aantal elementen bevatten:
\begin{center}
\begin{tabular}{ | c | c | c | c | c | }
\hline
 Datum & Afspraak & Wie & Gereed & Opmerkingen \\ 
\hline
\end{tabular}
\end{center}

Datum bevat de datum dat de afspraak gemaakt is en Gereed de datum waarop het gereed moet zijn. In de kolom met Afspraak komt te staan wat er afgesproken is en bij Wie de naam of namen van de personen die verantwoordelijk zijn voor het nakomen van de afspraak.



\chapter{Behoefte Analyse}

Voordat je een informatiesysteem gaat inrichten is het van belang om te weten wat een klant/gebruiker wil. De eerste taak die dan ook uitgevoerd moet worden is het verzamelen van informatie om te weten te komen wat de wensen van de gebruiker zijn. De belangrijkste twee vragen daarbij zijn de wat en de waarom vraag.

Wat een gebruiker wil is meestal al snel duidelijk. Een gebruiker wil bijvoorbeeld een website voor zijn winkel. Bij de wat vraag moeten ook de verwachtingen van de klant naar voren komen. Het maakt nogal wat uit of de klant 200 artikelen in zijn shop wil opnemen of 2.000.000. En het maakt voor de infrastructuur veel uit of hij 200 bestellingen per maand of 2.000.000 verwacht. Bij de waarom vraag moeten dus de doelen van de gebruiker naar boven komen.

Waarom een gebruiker het wil is een stuk lastiger, maar maakt uiteindelijk wel duidelijker waaraan het eindproduct moet voldoen. Een klant die een autoshowroom binnenkomt met zijn zwangere vrouw opzoek naar een nieuwe auto, daarvan is duidelijk dat ze geen behoefte hebben aan een cabriolet voor 2 personen. Waarom? Omdat er een kind bij komt, maar dit is een aanname en het is dus goed om de waarom vraag toch te stellen. Want misschien hebben ze al een 4-persoonsauto en zoeken ze een cabrio om er in de weekenden samen mee weg te kunnen als de baby bij opa en oma is. Zo zie je dat het belangrijk is om vragen altijd te stellen. Vragen stellen is niet dom, een vraag niet stellen en iets aannemen kan tot een verkeerd product en dus een ontevreden klant leiden.


\section{Stappenlan}
De behoefte analyse kent 6 stappen die doorlopen moeten worden.
\begin{enumerate}
\item Overleg met de opdrachtgever
\item Informatie verzamelen en vragen opstellen
\item Interviewen betrokkenen
\item Informatie controleren
\item Informatie structureren
\item Behoefte analyse vastleggen
\end{enumerate}

\section{Informatie verzamelen}
Bij het informatie verzamelen is het van belang om die informatie te verzamelen die van belang is voor het project. Zoek bijvoorbeeld op Internet naar de verschillende oplossingen die in de markt verkrijgbaar zijn en vergelijk de producten of oplossingen. Verzamel ook de informatie die al bekend is binnen de organisatie en zoek uit welke mensen er de kennis hebben om verdere vragen te kunnen beantwoorden, of welke mensen of organisatie onderdelen je nodig hebt om tot een oplossing te komen.

Het laatste onderdeel van het informatie verzamelen is het maken van afspraken met personen in of buiten de organisatie die je kunnen helpen om de vragen die je nog hebt te beantwoorden.

\subsection{Vragen stellen}
Zorg dat de opdracht duidelijk wordt. Stel daarvoor bijvoorbeeld de volgende vragen:
\begin{itemize}
\item Wat is de reden?
\item Wat is het doel?
\item Wat zijn de verwachtingen?
\item Welke werkzaamheden moeten worden uitgevoerd?
\item Wat is het budget?
\item Is er een tijdslimiet?
\item Wie hebben er belang bij de oplossing?
\item Aan welke personen kun je verdere vragen stellen?
\end{itemize}


\section{Doelbeschrijving}
Wat ook helder moet zijn voordat we aan een project beginnen is het doel. Een gebruiker heeft vaak de neiging om een product neer te leggen als oplossing. Ik wil Photoshop! Bij doorvragen komen er meestal eisen en wensen op tafel waaraan het gekozen product niet of niet optimaal voldoet. In de behoefte analyse is het dan ook van belang om nog niet in oplossingen te denken maar de informatie te verzamelen waaraan de oplossing moet voldoen. In het geval van Photoshop zou je dus moeten vaststellen wat de gebruiker wil doen:
\begin{itemize}
\item Moet foto's van het formaat TIFF, JPEG en DICOM kunnen lezen
\item Moet deze foto's kunnen roteren tussen 0 en 360 graden
\item Moet META-data aan foto's mee kunnen geven in IPTC en DICOM
\end{itemize}
Op deze manier maak je en lijst met wensen en eisen. Het is daarbij van belang om vast te stellen wat er in de eindoplossing moet zitten (eis) en wat fijn zou zijn als het erin zit (wens).

Een ander belangrijkpunt zijn de randvoorwaarden. Kortom binnen welke perken moetn we blijven qua budget, personen die ingezet mogen worden op het project of de tijdsduur, is er bijvoorbeeld een deadline waarvoor het project af moet zijn.

Het laatste dat we tijdens de behoefte analyse moeten inventariseren is wat de kansen en risico's zijn. Is het project bijvoorbeeld afhankelijk van een ander project, of kunnen we aansluiten bij een al bestaand project? Let ook op of er conflicten zijn tussen de verschillende wensen en eisen en of het geheel wel past binnen de bestaande infrastructuur. Het zou bijvoorbeeld zo kunnen zijn dat uit de wensen en eisen blijkt dat er een Linux met Apache-webserver moet komen, maar dat er binnen de organisatie alleen Windows gebruikt wordt. Het is dan de vraag of er geinvesteerd gaat worden in de opbouw van kennis om een Linux server te beheren, of dat er toch gekozen gaat worden voor een product dat kan draaien op een IIS server.

\section{Probleembeschrijving}
Om het probleem helder te krijgen beginnen we met vast te stellen wie de probleem-eigenaar is wat de beschrijving van het probleem is. We moeten daarbij de volgende vragen beantwoord zien te krijgen:
\begin{itemize}
\item Wie is de probleem-eigenaar?
\item Wat is het probleem?
\item Wie ervaart het probleem?
\item Wie heeft er belang bij?
\end{itemize}

\section{Interviewen}

Tijd is over het algemeen kostbaar, toch hebben we tijd nodig van mensen om beter inzicht te krijgen in de eisen en wensen van een organisatie. Om de hoeveelheid tijd die je van iemand vraagt te beperken is het noodzakelijk om je goed voor te bereiden. Zorg dat je vragen op papier staan, bedenkt wat de mogelijke antwoorden zouden kunnen zijn en bedenk dat je dan nog verder zou willen weten.

Weet aan wie je de vragen wil stellen. Wat is hun functie? En tot welke details zouden ze je antwoord kunnen geven? Een directeur weet vaak in grote lijnen wel bij welke afdeling je je vraag kwijt kan en die weet ook wat voor uitstraling hij wil hebben, maar die heeft vaak geen idee van welke server wat doet. Een systeembeheerder daarentegen weet precies welke server wat doet, maar heeft waarschijnlijk geen idee wat de bedrijfskleuren zijn en welk font er op een website gebruikt moet worden. Zo heeft iedereen binnen een organistatie zijn kennis en kunde, stel je vragen daarop af en kies de te interviewen mensen zorgvuldig.

Maak van elk gesprek dat je gevoerd hebt een gespreksverslag. Deel dit met de geinterviewde en vraag om een bevestiging. Alleen zo weet je zeker dat je de juiste informatie hebt opgeschreven en de persoon goed begrepen hebt. Fouten in het begin van het proces hebben grote gevolgen aan het eind. Als je twijfelt of je iets goed hebt genoteerd, vraag het dan opnieuw. Controle van je data is essentieel. Een niet gestelde vraag zorgt vrijwel zeker voor een niet goed opgeleverd project.

\section{MoSCoW}

Niet alles is even belangrijk in een project. Ook kunnen geld- of tijdslimieten ervoor zorgen dat zaken beter niet opgenomen kunnen worden. Het is dus van belang dat er vast gelegd gaat worden wat we wel en niet doen, en wat we doen als het kan. Binnen het bedrijfsleven is daarvoor de MoSCoW-methode ontstaan. MoSCoW is een afkorting waarbij de o's niet betekenen. Het gaat alleen om de letters MSCW

\begin{itemize}
\item M - Must have (moeten we hebben)
\item S - Should have (kunnen we eigenlijk niet zonder)
\item C - Could have (doen we als tijd en geld het toelaat)
\item W - Won't have (doen we niet)
\end{itemize}

Het is vaak moeilijk om de eisen en wensen in de juiste categorie te plaatsen.

De Must haves zijn echte eisen. In deze categorie komen zaken te staan die we moeten doen. Als deze zaken niet gedaan worden hebben we geen werkend product.

De Should haves zijn wensen die we moeten doen. Zonder deze wensen werkt het product wel, maar het voldoet dan niet aan bijvoorbeeld bedrijfs- of beveiligingsrichtlijnen. Als we niet aan deze wensen voldaan hebben kan het project niet in productie opgenomen worden

De Could haves zijn wensen die we willen doen, maar die niet verhinderen dat een project in productie gaat. Het zou mooi zijn als we ze kunnen vervullen en we zullen in het project ook ons best doen om het te doen, maar als ze niet gedaan zijn kan het project wel vast in productie gaan en worden ze op een later tijdstip alsnog gedaan.

Won't haves zijn zaken die we niet doen binnen het project. Het legt dus de grenzen vast. Dit is vaak het lastigste onderdeel om goed in te vullen.

\section{Template}
\begin{enumerate}
\item Inleiding
\item Opdracht
\item Aanleiding
\item Knelpunten
\item Eisen/wensen
	\begin{enumerate}
	\item Must haves
	\item Should haves
	\item Could haves
	\item Won't haves
	\end{enumerate}
\item Uit te voeren werkzaamheden
\end{enumerate}

In de inleiding leg je vast dat dit de behoefte analyse is en waar het onderwerp globaal over gaat.

In de opdracht beschrijf je wat de gekregen opdracht is in je eigen woorden en leg je het doel van het project vast.

Bij de aanleiding leg je vast waarom dit project wordt gedaan en wie er belang heeft bij het project. Dit legt onder andere vast wie er betrokkken moeten worden bij het project om tot een goed eindresultaat te komen.

Bij knelpunten leg je vast wat de problemen zijn waarvoor dit project de oplossing is. Het is belangrijk om aan het einde van het project te controleren of alle hier genoemde problemen ook daadwerkelijk opgelost zijn. Het kan heel handig zijn om hier een vinklijstje te maken van de bestaande problemen.

Bij uit te voeren werkzaameheden leg je een grof stappen plan vast. Je gaat niet in details maar geeft aan wat in welke volgorde moet gebeuren. Dit stappen plan legt ook vast wie er van de organistatie bij welke stap betrokken moet zijn. De organisatie moet hier zijn goedkeuring aan geven, want het betekent dat mensen de tijd moeten hebben om aanwezig te zijn, of om actie te ondernemen.


\chapter{Funtioneel Ontwerp}

In het functioneel ontwerp leg je in niet technische termen vast hoe het project eruit gaat zien. Dit document is over het algemeen voor management en zij moeten het document dan ook kunnen snappen. De opdrachtgever (klant) en de opdrachtnemer moeten hetzelfde beeld hebben van de uit te voeren opdracht.

Mocht je in een functioneel ontwerp het gebruik van technische termen niet kunnen vermijden zorg er dan voor dat je die termen uitlegt. Beter is het vaak om een plaatje te gebruiken, voor niet techneuten is dat vaak duidelijker dan tekst.

Flowcharts, stroomdiagrammen en globale netwerktekeningen zijn goede oplossingen om techniek uit te leggen aan niet techneuten.

Een functioneel ontwerp kan naast een document ook vaak leiden tot een presentatie waarin, door de opdrachtnemer, aan de opdrachtgever uitgelegd wordt wat het project betekent voor de organisatie en wat voor gevolgen het heeft.

\section{Stappenplan}
\begin{itemize}
\item Behoefte analyse omzetten naar functies in een informatie systeem
\item Gevolgen van de functies bepalen
\item Globale planning opstellen
\item Globale kostenraming opstellen
\item Afstemmen met opdrachtgever en collega's
\item Functioneel ontwerp vastleggen
\item Presentatie aan de belanghebbenden in de organisatie
\item Goedkeuring functioneel ontwerp
\end{itemize}

De eerste stap is het vertalen van de eisen en wensen naar functies. Een functie is een taak die uitgevoerd moet worden. Als de eis is dat er een tekst verwerkt moet kunnen worden dan kan een functie zijn het installeren van een tekstverwerker.

Door te bepalen wat de gevolgen zijn van een functie zet je de risico's voor een organisatie op een rij. Het installeren van een tekstverwerker betekent bijvoorbeeld dat er extra diskruimte in beslag genomen gaat worden en dat er tijdens het gebruik meer geheugen gebruikt gaat worden, maar het betekent bijvoorbeeld ook dat gebruikers hun documenten ergens moeten kunnen opslaan. Dit onderzoek naar de gevolgen kan behoorlijk complex zijn. Functies kunnen ook een invloed hebben op gebruikers (bijvoorbeeld training) of beheer (bijvoorbeeld up-to-date houden).

Op het moment dat je weet wat er allemaal gebeuren moet en wat dat voor gevolgen heeft kunnen een globale planning opzetten. Een planning is een lijstje dat aangeeft wat in welke volgorde gebeuren moet om tot de juiste oplossing te komen.

Een gevolg van een planning en een overzicht met wat er gebeuren moet is dat je een eerste kostenoverzicht kan maken. Je weet immers wat er aangeschaft moet worden (bijvoorbeeld hardware, of software) en wat er aan uren nodig zijn om tot een eindproduct te komen. Als het functioneel ontwerp goedgekeurd is kan je een offerte maken en ligt de prijs van het project definitief vast.

\section{Template}

\begin{enumerate}
\item Inleiding
\item Achtergrond informatie
\item Aannemer
\item Opdrachtgever
\item Probleemstelling
\item Functionele eisen
\item Nieuwe situatie
\item Globale planning
\item Globale kostenraming
\end{enumerate}

Bij de inleiding geef je aan dat dit het functionele ontwerp is voor het project met een minimale omschrijving van waar het project over gaat. Ook geef je hierin aan voor wie dit document bedoelt is.

In de achtergrond informatie beschrijf je het bedrijf of de organisatie waarvoor het project wordt uitgevoerd. Wat voor bedrijf is het? Hoeveel mensen werken er? Wat is de huidige situatie? en waarom gaan we dit project doen?

De aannemer is het bedrijf, organisatie of persoon die de opdracht gaat uitvoeren. Naast de naam en contactgegevens kan hier een stukje omschrijving geplaatst worden over de aannemer.

De opdrachtgever is het bedrijf of de afdeling binnen een bedrijf waarvoor we de opdracht uitvoeren. Neem hierin ook de contact gegevens op van de mensen binnen de organisatie die belangrijk zijn voor het slagen van het project.

De probleemstelling bevat een uitgebreidde omschrijving van welk probleem het project moet oplossen.

De functionele eisen leggen vast wat er moet gebeuren en hier kan de MoSCoW-methode weer gebruikt worden

Tot slot leg je vast in de nieuwe situatie wat het eindresultaat zal zijn en wat dat voor gevolgen heeft voor de organistatie.

Voor het management volgen hierna nog een grove planning, zodat duidelijk is hoelang een project gaat duren (en waarom) en een globaal overzicht van de kosten.

\section{Beoordeling}

Het is van essentieel belang dat een functioneel ontwerp goedgekeurd wordt door een organisatie. Het functioneel ontwerp kan de basis zijn voor een offerte, maar vormt zeker de basis voor het technische ontwerp dat vast legt wat er exact gaat gebeuren en hoe. Als er geen overeenstemming is over het functioneel ontwerp kan een project dus niet verder. Alle betrokken partijen moeten hun goedkeuring geven zodat later niet gezegd kan worden dat een project niet naar behoren is uitgevoerd.

\section{Offerte}

Als iedereen het erover eens is wat er moet gebeuren dan weten we wat er nodig is aan materiaal en kunnen we berekenen wat er aan uren nodig is om het project uit te voeren. Meestal gaat er binnen grote bedrijven nu een calculator aan de slag om het project vollledig door te rekenen. Op basis van deze doorrekening kan een offerte gemaakt worden. Na het tekenen van de offerte wordt het technisch ontwerp geschreven en gaat het projectteam aan de slag om de oplossing te realiseren.



\chapter{Technisch Ontwerp}

In het technisch ontwerp maak je de vertaling van de functionaliteit naar een daadwerkelijk product of dienst die je gaat realiseren. Als de functie een tekstverwerker was dan komt er in het technisch ontwerp te staan dat er bijvoorbeeld LibreOffice Writer versie 7 wordt ge\"installeerd. Een technisch ontwerp gaat dus in op de technische details van een project en mag, of moet, de technische termen bevatten. Tekeningen gaan verder in op details, in plaats van wolkjes voor een netwerk worden de daadwerkelijke verbindingen getekend, etc.

Het is belangrijk om alle details te verwerken in het technisch ontwerp. Feitelijk moet het zo zijn dat als een ander het technisch ontwerp volgt dat hij dan een identieke installatie oplevert.

In het technisch ontwerp staat ook hoe de gekozen oplossing aansluit bij de bestaande infrastructuur en welke technische gevolgen dat heeft.

Het technisch ontwerp is bedoelt voor de technisch beheerders van de infrastructuur van de klant zodat deze weten wat ze krijgen en wat er van hun verwacht wordt, maar het is ook voor het projectteam zodat ze weten wat ze wanneer moeten opleveren.

\section{Stappenplan}

\begin{itemize}
\item omzetten van de functies in het functioneel ontwerp naar daadwerkelijke producten of diensten
\item Koppeling met bestaande systemen beschrijven
	\begin{itemize}
	\item gedetaileerde beschrijving van hoe de producten of diensten aansluiten op de bestaande infrastructuur
	\item beschrijving van wat er nodig is om de producten of diensten te kunnen plaatsen binnen de huidige infrastructuur
	\end{itemize}
\item Selecteren van systemen, producten, software
	\begin{itemize}
	\item welke producten of diensten worden er geleverd
	\end{itemize}
\item Afstemming met interne (klant) en eigen collega's
	\begin{itemize}
	\item beschrijving van de wijzigingen in de infrastructuur of in de dienstverlening en de impact op de organisatie (beheer, support)
	\end{itemize}
\item 
\end{itemize}

De eerste stap is het technisch beschrijven van de producten of diensten die worden geleverd voor elke functie uit het functioneel ontwerp. De installatie van een tekstverwerker wordt daadwerkelijk gerealiseerd met LibreOffice Writer versie 7.

Op basis van de te leveren diensten of producten moeten er systemen uitgezocht worden die de technische diensten gaan waar maken. Voor de opslag van de gebruikers documenten moet er op de fileserver per gebruiker een directory aangemaakt worden om bestanden op te kunnen slaan. Dit heeft tot gevolg dat het RAID systeem van de server met twee extra harddisks moet worden uitgebreid. De harddisks moeten 10k rpm harddisks zijn die passen binnen de bestaande RAID set met 600GB schijven gebaseerd op een Dell PERC H710 Controler.

Zorg ook dat bijvoorbeeld IP-adressen, systeemnamen en ook lokaties in kasten bekend zijn. Zijn er voldoende netwerkpoorten en spanningsaansluitingen, etc. Denk daarbij ook aan printers, werkplekken en randapparatuure.

Naast LibreOffice Writer 7 moeten er dan bijvoorbeeld ook Dell 600GB 10K RPM SAS 12Gbps schijven worden aangeschaft. Dit komt in een lijst te staan met te leveren producten.

Tot slot is het van belang dat iedereen, zowel de beheerders van de klant als de leden van het projectteam weten wat er wanneer van ze wordt verwacht. Wanneer kunnen bijvoorbeeld de harddisks in de server geplaatst worden en welke personen van de klant moeten daarbij aanwezig zijn voor toegang tot de serverruimte. Let er ook op dat na de implementatie de organisatie het systeem moet kunnen beheren en supporten, dus ook daarvoor moeten de zaken op orde zijn (kennis, documentatie).

\section{Template}

\begin{enumerate}
\item Inleiding
\item De oplossing in kaart
	\begin{enumerate}
	\item Beschrijving van de oplossing
	\item Plaats van de oplossing in de infrastructuur
	\item Technische randvoorwaarden
	\end{enumerate}
\item Overzicht technische opbouw
\item Technische uitwerking
\item Applicatie configuratie
	\begin{enumerate}
	\item Installatie
	\item Configuratie- / systeeminstellingen
	\end{enumerate}
\item Beveiliging
\end{enumerate}

In de inleiding geef je aan waarom dit technische ontwerp er is. Vertel daarbij iets over de opdracht en de huidige situatie. De doelgroep is de beheerders, dus het mag technisch zijn.

Bij de beschrijving van de oplossing beschrijf je hoe de onderdelen van het functieoverzicht uit het functioneel ontwerp uitgevoerd gaan worden.

Beschrijf hoe de nieuwe omgeving onderdeel gaat worden van de bestaande (netwerk)infrastructuur, plaats dat onder het kopje Plaats van de oplossing in de infrastructuur.

Geef een overzicht van welke technische zaken geregeld moeten zijn voordat begonnen wordt met het maken van de nieuwe of vernieuwde omgeving. Dit overzicht is belangrijk voor de technische randvoorwaarden.

In het overzicht technische opbouw geef je een gedetailleerde beschrijving van hoe de nieuwe of vernieuwde omgeving opgebouwd gaat worden. Geef hierbij duidelijk aan welke technieken en systemen er gebruikt worden. Zorg ook voor netwerkplaatjes en overzichten zodat duidelijk is hoe het nieuwe netwerk eruit komt te zien.

In de technische uitwerking beschrijf je per onderdeel welke vernieuwingen en/of aanpassingen doorgevoerd worden, denk daarbij aan: werkstations, servers, routers, switches en printers.

Zorg ervoor dat de installatie goed beschreven is, zodat dit herhaald kan worden en dat er exact dezelfde oplossing uit komt.

Voor de configuratie geldt hetzelfde. Elke wijziging in de configuratie moet zo gedocumenteerd zijn, dat elke stap gedocumenteerd is.

Tot slot documenteer je de volledige security die er is geregeld. Denk daarbij aan:
\begin{itemize}
\item Toegangsbeveiliging
\item Rechten
\item Groepen
\item Inbraakbeveiliging
\item Beveiliging tegen malware
\item Back-up/restore
\item Beveiligde verbinding
\end{itemize}

\section{Akkoord}

Het uitvoeren van een project heeft een impact op de bestaande omgeving. Zorg er dus voor dat je toestemming hebt voor de verschillende onderdelen uit het project om ze te mogen uitvoeren.


\chapter{Testplan}
Voordat een project opgeleverd kan worden moet er getest worden. Hier komt het MoSCoW overzicht weer om de hoek kijken. De te testen onderdelen zijn die zaken die beschreven zijn in de Must haves en Should haves. Natuurlijk moeten ook de zaken getest worden die in de Could haves beschreven staan en ook daadwerkelijk ge\"implementeerd zijn.

\section{Stappenplan}
Het testen van een systeem kent een aantal fases:
\begin{enumerate}
\item Tests beschrijven (Testplan)
\item Planning maken (Testplan)
\item Testomgeving maken en tests uitvoeren
\item Verloop van de tests vastleggen en uitkomsten verwerken (Testrapport)
\end{enumerate}

\subsection{Tests}
Bij de tests beschijf je wat je gaat doen om een onderdeel van het systeem te testen. Beschrijf wat je gaat testen:
\begin{itemize}
\item functie uit een functioneel ontwerp
\item een regel uit een beleidsplan
\item iets anders
\end{itemize}

Beschrijf hoe je gaat testen, denk daarbij aan:
\begin{itemize}
\item Wat heb je nodig aan hardware?
\item Wat heb je nodig aan software?
\item Wat heb je nodig aan mensen?
\item Wat is je input data?
\end{itemize}

Beschrijf ook wat je verwacht dat het resultaat is van de test.

\section{Templates}
\subsection{Template IEEE 829}

Volgens de IEEE 829-2008 (Standard for Software and System Test Documentation) moet een testplan de volgende layout hebben:
\begin{itemize}
\item Pagina met daarop: Naam van het testplan en versienummer
\end{itemize}

\begin{enumerate}
\item Inleiding
\item Onderdelen die getest worden
\item Onderdelen die niet getest worden
\item Aanpak
\item Criteria voor pass of fail van een test
\item Criteria voor het onderbreken van een test en wanneer de test hervat wordt
\item Test uitkomsten
\item Test taken
\item Randvoorwaarden
\item Verantwoordelijkheden
\item Personeel en opleidingsbehoefte
\item Planning
\item Risico's
\item Goedkeuringen
\end{enumerate}

Het doel en wat we willen bereiken met de testen wordt beschreven in de inleiding. Hier moet ook vastliggen wat limieten zijn zowel qua geld als qua andere middelen zoals bijvoorbeeld benodigde tijd en personeel moet in de inleiding beschreven zijn.

De wel of niet te testen onderdelen en functies worden beschreven in de hoofdstukken Onderdelen die getest en Onderdelen die niet getest worden. In deze hoofstukken moeten referenties aanwezig zijn naar bijvoorbeeld het functioneel ontwerp of de behoefte analyse zodat duidelijk is dat de testen aansluiten of de behoefte van de organisatie.

In het hoofdstuk over de aanpak beschrijf hoe er getest gaat worden. Per test item is het noodzakelijk om vast te leggen welke data wordt gebruikt als input en output, wat de test procedures zijn per item en wat de volgorde is waarin de items getest moeten worden. Het is ook belangrijk om hier vast te leggen er nodig is om de testen te kunnen doen.

Bij de Test uitkomsten leg je vast wat er uit de test moet komen om succesvol te zijn, maar ook wat er zou moet gebeuren als een test faalt. Kan een volgende test gedaan worden, of is het een no-go criterium. De reden om het testen te staken of juist weer te hervatten zal per item vastgelegd moeten worden.

Wie er verantwoordelijk is voor welke taak op welk moment wordt vastgelegd in het Test taken hoofdstuk. Ook zal er per taak vastgelegd moeten worden wat ervoor nodig is en wat de doorlooptijd is. Dit hoofdstuk bevat dus de volledige planning voor de tests. Per taak zal er beschreven moeten worden wat er gedaan wordt zodat de verantwoordelijke voor test zijn taak goed kan uitvoeren. De verantwoordelijk zal ook een log van zijn testen bij moeten houden gedurende het testen zodat elke stap van de test naderhand gecontroleerd kan worden. Leg per test ook vast wat er opgeleverd moet worden naast de log om aan te tonen dat een test geslaagd is (of niet). De op te leveren gegevens zouden kunnen zijn, output data, screenshots, logs, etc.

Randvoorwaarden beschrijft welke zaken er aanwezig en op orde moeten zijn voordat aan een test begonnen kan worden. Het kan dan gaan om hardware zaken zoals spanning en netwerkaansluitingen, maar ook om personele aanwezigheid of toegang tot ruimtes. Beschrijf hier ook wat er nodig is als er een speciale testomgeving gebouwd moet worden.

Verantwoordelijkheden bevat de taakomschrijvingen van het testteam, ervan uitgaande dat er meer dan 1 persoon betrokken is bij de test.

Als er mensen opgeleid moeten worden of speciaal geinstrueerd moeten worden voordat een test plaats kan vinden dan wordt dat beschreven in Personeel- en opleidingsbehoefte.

Een overzicht van welke test op welke datum en op welk tijdstip moet worden uitgevoerd door wie wordt vastgelegd in de Planning. Gebruik hiervoor een applicatie om planningen te maken zoals ProjectLibre.

Aan testen kunnen risico's verbonden zijn, zeker als bij de testen productie systemen betrokken zijn. Het is belangrijk belangrijk dat deze risico's beschreven zijn en dat er vastgelegd is hoe de risico's geminimaliseerd kunnen worden en wat te doen als er iets fout mocht gaan.

Uiteindelijk moet je toestemming krijgen om de testen te mogen uitvoeren. In Goedkeuringen staan de namen van de personen die moeten tekenen om goedkeuring te geven aan de testen. Pas als alle handtekeningen binnen zijn kan er aan het testen begonnen worden.

\subsection{Template SPL}

\renewcommand{\labelenumii}{\theenumii}
\renewcommand{\theenumii}{\theenumi.\arabic{enumii}.}
\renewcommand{\theenumiii}{\theenumii\arabic{enumiii}}

Bij het examen gebruikt Stichting Praktijkleren een andere variant op het testplan. Deze heeft de volgende structuur:
\begin{enumerate}
\item Inleiding
\item Overzicht van te testen functies
	\begin{enumerate}
	\item Functie x
		\begin{enumerate}
		\item Te testen functionaliteit
		\item Test scenario
		\item Test input
		\item Verwachte werking en output
		\item Werkelijke werking en output
		\item Conclusie test
		\end{enumerate}
	\end{enumerate}
\item Test omgeving
	\begin{enumerate}
	\item Overzicht test omgeving
	\item Planning opzet test omgeving
	\end{enumerate}
\item Planning testactiviteiten
\item Eind conclusie testen
\end{enumerate}

Het testplan en het testrapport zijn in deze opzet in \'e\'en document verwerkt.

In het Overzicht van te testen functies beschrijf je per functie hoe de test eruit moet zien en is er tevens ruimte om de uitkomsten van de test te verwerken. De x bij item 1.1 moet vervangen worden door de naam van de functie die getest gaat worden.

Bij Test omgeving beschrijf je wat er nodig is om de test te kunnen uitvoeren. Dit is het materiaal dat nodig is, de software en de mensen. Het bevat dus ook een planning die nodig is om de testomgeving te realizeren.

Om de testen die beschreven zijn te kunnen uitvoeren moet er een planning gemaakt worden waarin beschreven wordt wie welke testen uitvoert. Het kan belangrijk zijn om de testen in een bepaalde volgorde te doen, als dat zo is moet dat uit de planning blijken. De planning bevat de startdatum en het tijdstip waarop de test uitgevoerd wordt. Daarnaast legt het ook vaak vast hoelang een test duurt of voor wanneer hij klaar moet zijn.

Het laatste onderdeel van dit testplan is de Eind conclusie testen. Bij de conclusie beschrijf je of de totale test een uitkomst die voldoende was om het project als succesvol te beschouwen, of je geeft aan wat er niet goed gegaan is en wat er nodig is om verdere stappen te ondernemen.


\section{Het testen}
Gedurende de testen is het noodzakelijk om een log bij te houden en eventueel screenshots te maken. Het is belangrijk om het tijdsverloop vast te leggen om aan te kunnen tonen dat de tests binnen het gestelde tijdspad zijn gebeurd. Door regelmatig vast te leggen wat de tussen resultaten van een test zijn kunnen later betere analyses gegeven worden van de resultaten. Zeker als een test niet lukt is het goed om te weten op welk punt het mis gegaan is.

Uiteindelijk moet uit de tests blijken dat het systeem voldoet aan de in de behoefte analyse beschreven wensen en eisen.

\section{Testrapport}
De uitkomsten uit de gedane testen moeten vastgelegd worden in een testrapport. Dit rapport wordt voorgelegd aan de organisatie ter goedkeuring. Als de organisatie akkoord gaat met de uitkomsten van de testen kan het systeem in productie genomen worden.

Worden de uitkomsten niet geaccepteerd, dan zal men moeten kijken wat de vervolgstappen worden:
\begin{itemize}
\item Verbeteringen doorvoeren om tot andere testuitkomsten te komen
\item Wijzigingen in de testen doorvoeren omdat de gekozen testen niet correct zijn
\item Terug naar de tekentafel om tot een ander ontwerp te komen dat wel aan de eisen van de opdrachtgever voldoet
\item Het project in zijn geheel afblazen
\end{itemize}

\subsection{Testresultaten}
Per test beschrijf je wat de uitkomst is van de test en of dit overeenkomt met de verwachtte resultaten. Als de uitslag niet overeenkomt met het verwachtte resultaat geef dan een beschrijving waarom het resultaat niet overeenkomt met de verwachting. Geef ook een aanbeveling wat er gedaan kan worden om ervoor te zorgen dat de test wel slaagt, dit hoeven niet alleen technische aanbevelingen te zijn.

\subsection{Conclusie}
Het is handig om in een testrapport ook een conclusie te schrijven over de uitslagen van het testproces. Dit is dan een beknopte managementsamenvatting, die zo simpel kan zijn als: Alle testen zijn gedaan en de uitkomsten voldeden aan de gestelde criteria.

\subsection{Goedkeuring}
Tot slot dient er goedkeuring verkregen worden voor de testresultaten zodat besloten kan worden of het project succesvol verlopen is.


%%%%%%%%%%%%%%%%%%%%%
%%% Index and End %%%
%%%%%%%%%%%%%%%%%%%%%
\backmatter
\printindex
\end{document}

%%% Last line %%%
