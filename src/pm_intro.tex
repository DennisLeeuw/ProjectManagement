In de opdracht van de overheid wat een MBO student moet kunnen als hij van school komt staat dat deze instaat moet zijn een informatievoorziening te ontwerpen. Een informatievoorziening kan een webserver zijn, maar ook een webserver met database of een compleet netwerk met authenticatiesystemen, opslagsystemen en gebruikerssystemen. Een informatievoorziening of een informatiesysteem is een samenhangend geheel van apparaten die data verwerken. En dat moet je als MBO-er dus kunnen ontwerpen.

Een informatievoorziening is dus een systeem dat gebouwd moet worden van begin tot eind. Vanuit de ITIL weten we dat als we een actie hebben met een duidelijk begin en een duidelijk einde dat we dan spreken van een project. Om een project in goede banen te leiden moeten we een project beheren of managen. Vandaar dat dit document gaat over project management. Je leert welke stappen je moet doorlopen om een project van begin tot eind te doorlopen zodat de wensen van een klant of gebruiker gerealizeerd worden en er een gelukkige klant is.

Om te zorgen dat een informatiesysteem opgeleverd wordt volgens de eisen en wensen van de klant is het van belang om een aantal stappen te doorlopen. De eerste stap is dat het duidelijk moet zijn wat de klant wil. Daarna kan er een plan gemaakt worden en kunnen we aan de bouw beginnen. Tot slot moet het gebouwde systeem opgeleverd worden en hopen we dat we een gelukkige klant hebben. Dit zijn de stappen die in bijna elk project doorlopen moeten worden om te komen tot een werkende oplossing. Projectmanagement (ook wel projectbeheer) is het beheersen van projecten. Het is de manier waarop projecten georganiseerd, voorbereid, gepland, uitgevoerd en afgerond worden. Er is een functie voor deze taak, dat zijn de zogenaamde projectmanagers. Deze managers doen niets anders dan projecten, dat werkt goed binnen grote organisaties, binnen kleine(re) organisaties wordt het projectmanagement vaak gedaan door degene die het meeste belang heeft bij het project.

In de loop van de tijd zijn er verschillende visies ontstaan op hoe al deze stappen doorlopen moeten worden. Algemene projectmanagement methodes zijn:
\begin{itemize}
\item EVO
\item IPMA
\item OPEN
\item PRINCE2
\item PMBOK
\item PM$^2$
\item Het PM3-model
\end{itemize}

In de ICT zijn verschillende methodes ontstaan voornamelijk rond software-ontwikkeling die veel gebruikt worden:
\begin{itemize}
\item Agile
\item DSDM
\item RUP
\item Scrum
\item SDM
\item Watervalmethode
\end{itemize}

De verschillende methodes zijn soms vrij beschikbaar voor iedereen, soms zijn ze gecertificeerd, wat inhoudt dat gebruikers een certificaat/diploma moeten halen om zichzelf 'gecertificeerd' projectleider te mogen noemen. De certificaten worden in de regel verkregen door (tegen betaling) een examen te doen bij een instituut dat de desbetreffende methode heeft uitgevonden. Dit certificaat moet in de regel om de zoveel jaar vernieuwd worden.

De waterval methode is behoorlijk formeel, met veel documentatie en is niet erg flexibel, als de plannen vastliggen is er geen wijziging meer mogelijk tijdens het bouwproces. Bij de waterval methode worden alle stappen \'e\'en voor \'e\'en doorlopen en wordt zo naar de eindoplossing toegewerkt.

De agile en scrum zijn wat losser, met minder documentatie en iets flexibeler, er kunnen wijzigingen blijven plaats vinden gedurende het bouw proces. Bij beide methodes wordt \'e\'en stap doorlopen en deze wordt geanalyzeerd en er wordt gekeken of we nog op het doel af gaan, of dat door nieuwe inzichten het doel verschoven is. Zodat we moeten bijsturen voordat we de volgende stap zetten. Het is dus een meer iteratief proces.

Welke methode het beste is hangt af van het product of de oplossing die binnen het project gerealizeerd moet worden en het hangt heel vaak ook af van de organisatie waarvoor je werkt. In dit document zullen we alle stappen op een waterval manier doorlopen. Er zijn een aantal redenen waarom er gekozen is voor de watervalmethode. De eerste is dat we waterval methode duidelijk afgebakende stappen kent. De tweede reden is dat het terug komt op je examen. En de laatste reden is dat als je de watervalmethode kent, je makkelijk andere methodes ook begrijpt.
